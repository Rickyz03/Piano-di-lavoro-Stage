%----------------------------------------------------------------------------------------
%	STAGE DESCRIPTION
%----------------------------------------------------------------------------------------
\section*{Scopo dello stage}
% Personalizzare inserendo lo scopo dello stage, cioè una breve descrizione
Lo stage ha come obiettivo l'acquisizione di competenze avanzate nello sviluppo di soluzioni di Intelligenza Artificiale applicate all’analisi automatica di dati aziendali.
Lo studente si occuperà di progettare e realizzare, utilizzando la piattaforma Oribea (basata su DialogSphere), un Task AI per l’analisi di bilancio partendo da dati estratti da database aziendali o da dataset pubblici (es. Kaggle).

L’attività prevede l’uso di Large Language Models (LLMs) integrati tramite LangChain, con l’obiettivo di automatizzare la generazione di report e insight aziendali.
Il progetto consentirà allo studente di acquisire esperienza pratica nello sviluppo di soluzioni AI per contesti reali, con focus su data analysis, integrazione LLM e pipeline di automazione.

Lo studente avrà il compito di:
\begin{itemize}
 \item Analizzare dataset aziendali e/o pubblici utili allo scopo;
 \item Sperimentare l’uso della piattaforma Oribea, che integra il framework DialogSphere;
 \item Progettare e sviluppare un prototipo in Python che, con il supporto di Large Language Models (LLMs) e strumenti di Machine Learning, produca analisi automatiche, interpretabili e personalizzabili;
 \item Documentare le fasi progettuali, le soluzioni tecniche adottate e le criticità affrontate.
\end{itemize}