%----------------------------------------------------------------------------------------
%   USEFUL COMMANDS
%----------------------------------------------------------------------------------------

\newcommand{\dipartimento}{Dipartimento di Matematica ``Tullio Levi-Civita''}

%----------------------------------------------------------------------------------------
% 	USER DATA
%----------------------------------------------------------------------------------------

% Data di approvazione del piano da parte del tutor interno; nel formato GG Mese AAAA
% compilare inserendo al posto di GG 2 cifre per il giorno, e al posto di 
% AAAA 4 cifre per l'anno
\newcommand{\dataApprovazione}{15 Aprile 2025}

% Dati dello Studente
\newcommand{\nomeStudente}{Riccardo}
\newcommand{\cognomeStudente}{Stefani}
\newcommand{\matricolaStudente}{2068225}
\newcommand{\emailStudente}{riccardo.stefani.10@studenti.unipd.it}
\newcommand{\telStudente}{+ 39 338 912 5689}

% Dati del Tutor Aziendale
\newcommand{\nomeTutorAziendale}{Marco}
\newcommand{\cognomeTutorAziendale}{Macari}
\newcommand{\emailTutorAziendale}{marco.macari@oribea.ai}
\newcommand{\telTutorAziendale}{+ 39 335 738 4815}
\newcommand{\ruoloTutorAziendale}{CEO} % Mai utilizzato

% Dati dell'Azienda
\newcommand{\ragioneSocAzienda}{Oribea AI S.r.l.}
\newcommand{\indirizzoAzienda}{Via Tre Settembre 99, 47891 Dogana, San Marino}
\newcommand{\sitoAzienda}{https://www.oribea.ai/}
\newcommand{\emailAzienda}{info@oribea.ai}
\newcommand{\partitaIVAAzienda}{P.IVA 12345678999} % Utilizzato solo nello stile aziendale, che io però non uso, quindi non serve

% Dati del Tutor Interno (Docente)
\newcommand{\titoloTutorInterno}{Prof.}
\newcommand{\nomeTutorInterno}{Marco}
\newcommand{\cognomeTutorInterno}{Zanella}

\newcommand{\prospettoSettimanale}{
     % Personalizzare indicando in lista, i vari task settimana per settimana
     % sostituire a XX il totale ore della settimana
    \begin{itemize}
        \item \textbf{Prima Settimana - Introduzione (40 ore)}
        \begin{itemize}
            \item Introduzione alla piattaforma Oribea e DialogSphere;
            \item Ripasso di Python, Pandas, LangChain;
            \item Studio delle funzionalità principali e delle best practice per lo sviluppo di Task AI.
        \end{itemize}
        \item \textbf{Seconda Settimana - Recupero dei dati (40 ore)} 
        \begin{itemize}
            \item Connessione ai database;
            \item Esplorazione dei dataset aziendali e pubblici;
            \item Definizione casi d’uso.
        \end{itemize}
        \item \textbf{Terza Settimana - Esplorazione dei dati (40 ore)} 
        \begin{itemize}
            \item Analisi dati;
            \item Sviluppo di funzioni di preprocessing;
            \item Prime prove con LLM e LangChain.
        \end{itemize}
        \item \textbf{Quarta Settimana - Esperimenti (40 ore)} 
        \begin{itemize}
            \item Progettazione e implementazione del primo Task AI per l’analisi di bilancio automatica.
        \end{itemize}
        \item \textbf{Quinta Settimana - Miglioramento (40 ore)} 
        \begin{itemize}
            \item Ottimizzazione dei prompt e output LLM;
            \item Test del sistema.
        \end{itemize}
        \item \textbf{Sesta Settimana - Integrazione (40 ore)} 
        \begin{itemize}
            \item Integrazione del prototipo con strumenti aziendali;
            \item UX di base;
            \item Miglioramento del flusso di utilizzo.
        \end{itemize}
        \item \textbf{Settima Settimana - Funzionalità avanzate (40 ore)} 
        \begin{itemize}
            \item Introduzione di funzionalità avanzate (es. explainability, filtri dinamici, personalizzazione).
        \end{itemize}
        \item \textbf{Ottava Settimana - Conclusione (40 ore)} 
        \begin{itemize}
            \item Test finali;
            \item Completamento documentazione tecnica;
            \item Raccolta di materiali utili ai fini della tesi.
        \end{itemize}
    \end{itemize}
}

% Indicare il totale complessivo (deve essere compreso tra le 300 e le 320 ore)
\newcommand{\totaleOre}{320}

\newcommand{\obiettiviObbligatori}{
	 \item \underline{\textit{O01}}: Acquisizione di competenze pratiche su Oribea/DialogSphere;
	 \item \underline{\textit{O02}}: Connessione a database e gestione dati aziendali o pubblici;
	 \item \underline{\textit{O03}}: Implementazione di un Task AI per l’analisi di bilancio con LLM;
	 \item \underline{\textit{O04}}: Generazione automatica di report con output coerente, chiaro e adattabile;
	 \item \underline{\textit{O05}}: Testing e documentazione completa del prototipo.
	 
}

\newcommand{\obiettiviDesiderabili}{
	 \item \underline{\textit{D01}}: Ottimizzazione del Task AI per performance e scalabilità;
	 \item \underline{\textit{D02}}: Personalizzazione dinamica dei prompt per casi d’uso differenti;
	 \item \underline{\textit{D03}}: Integrazione con strumenti di visualizzazione o interfacce utente.
}

\newcommand{\obiettiviFacoltativi}{
	 \item \underline{\textit{F01}}: Sviluppo di una semplice interfaccia chatbot o dashboard interattiva;
	 \item \underline{\textit{F02}}: Sperimentazione di tecniche di Explainable AI (XAI) per la trasparenza dei risultati;
	 \item \underline{\textit{F03}}: Esportazione automatica dei report in PDF/HTML o invio via e-mail.
}